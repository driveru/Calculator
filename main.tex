\documentclass[12pt,a4paper]{article}

\usepackage{cmap}
\usepackage[T2A]{fontenc}
\usepackage[utf8]{inputenc}
\usepackage[english,russian]{babel}
\usepackage{multicol}
\usepackage{graphicx}
\graphicspath{ {./} }
\usepackage{lipsum}
\usepackage{float}
\usepackage{verbatim}
\usepackage{setspace}
\usepackage{array}
\usepackage{amssymb,amsmath,amsthm,enumitem}
\setstretch{0.83}
\usepackage[
    top=2cm,bottom=2cm,bindingoffset=0cm]{geometry}
\usepackage{caption}
\setcounter{figure}{2}

\begin{document}
\thispagestyle{empty}
\setlength{\abovedisplayskip}{3pt}
\setlength{\belowdisplayskip}{3pt}
\captionsetup[figure]{labelfont=it,textfont=it,singlelinecheck=off,justification=raggedright}

\begin{multicols}{2}
\noindent тока) определяется из уравнения
$$\frac{1}{2}Mv^2_{max} = \omega_{max}L$$
и составляет
\begin{equation}\label{eq8}
v_{max} = \sqrt{2\omega_{max}L/M}. \tag{8}
\end{equation}   
При заданном значении $v_{max}$ минимальная длина веревки должна быть
\begin{equation}\label{eq9}
L_{min} = \frac{1}{2}\frac{Mv^2_{max}}{\omega_{max}} \tag{9}
\end{equation}   
Далее мы используем простое приближение (4). При этом энергия растянутой веревки ${W = F^2L/2S_0E_{\text{ср}}}$ (см.~формулы (2),~(3)). Приравнивая это выражение к кинетической энергии судна, получаем, что при скорости течения ${v<v_{max}}$ максимальное натяжение веревки есть
\begin{equation}\label{eq10}
F_{v} = F_{max}\frac{v}{v_{max}}. \tag{10}
\end{equation}

Ну а теперь выясним, какова роль силы сопротивления воды. Эта сила максимальна, когда судно остановилось, и при скорости потока $v$ составляет ${F_{\text{с}}=Mv^2/l_{\text{эф}}}$

Найдем отношение этой добавки к натяжению веревки $F_v$. С учетом соотношений (3), (\ref{eq8}) и (\ref{eq10}),
\begin{scriptsize}
    \begin{center}
    \begin{align*}
        \frac{F_c}{F_v}=\frac{Mv^2}{l_{\text{эф}}F_v}=\frac{Mv^2}{l_{\text{эф}}F_{max}(v/v_{max})}\cdot\frac{Mv^2_{max}/2}{Mv^2_{max}/2} =\\
        =\frac{v^2}{l_{\text{эф}(v/v_{max})}}\cdot\frac{2LF_{max}\varepsilon_{max}/2}{v^2_{max}F_{max}} =\\
        =\frac{L\varepsilon_{max}}{l_{\text{эф}}}\cdot\frac{v}{v_{max}}.
    \end{align*}
    \end{center}
\end{scriptsize}
\noindent
Таким образом, роль сопротивления воды описывается параметром
$$\beta=L\varepsilon_{max}/l_{\text{эф}}.$$

Обычно при таком зачаливании используется веревка не длинее 20~м. При $\varepsilon_{max} = 0.15$ это дает полное удлинение веревки 3~м, и при $l_{\text{эф}}\gtrsim10\text{ м}$
$$\beta\lesssim1/3. $$

Итак, при скорости течения меньше $v_{max}$ величина $F_{\text{с}}$ не превосходит $F_{max}/3$. Поэтому добавка, обусловленная давлением потока, играет небольшую роль, и предельная скорость близка к $v_{max}$.

Ну а теперь надо дать численные оценки для туристов. Пусть судно необходимо остановить на струе при скорости течения ${4 \text{ м/с} = 14 \text{ км/ч}}$ (это большая скорость!).

Необходмые параметры для веревки мы определим из формул (\ref{eq9}), (3) и (6). Чтобы учесть разброс параметров и давление потока, мы подставим в формулу (\ref{eq9}) несколько большее значение $v_{max} = 5\text{ м/с}$. Результаты расчетов для судов разных масс приведены в таблице, где указана (округленно) минимальная длина чальной веревки при различных типовых значениях ее диаметра. Это, собственно, и есть главный результат.

\begin{footnotesize}
\begin{center}
\setstretch{1}
\begin{tabular}{|m{0.2\columnwidth} |m{0.68\columnwidth}|}
\hline
\vspace{5pt}\centering{Масса судна}\vspace{5pt}& 
\vspace{5pt}Диаметр веревки и ее минимальная длина\vspace{5pt}\\
\hline
\vspace{5pt}\begin{tabular}{r}
200 кг \\
400 кг \\
800 кг \\
1200 кг\vspace{5pt}
\end{tabular} &
\vspace{5pt}\begin{tabular}{l}
6 мм, 10 м\\
6 мм, 20 м или 8 мм, 10 м\\
8 мм, 20 м или 10 мм, 10 м\\
10 мм, 20 м\vspace{5pt}\\
\end{tabular}\\
\hline
\end{tabular}
\end{center}
\end{footnotesize}
\setstretch{0.83}

И в заключение рассмотрим еще один вариант зачаливания.

\textbf{Поперечная веревка.}
Иногда для остановки судна поперек реки навешивают веревку. Конец чальной веревки с идущего судна с помощью крюка цепляется за эту поперечную веревку.
Ее навешивают под углом к течению, чтобы судо соскользнуло вдоль нее к берегу, или предусматривает ее обрыв с одного берега так, чтобы судно <<маятником>> пришло к другому берегу.

Если эти варианты реализовать нельзя, веревку навешивают поперек течения. При натяжении чальной веревки натягивается и эта поперечная веревка. Ее натяжение возьмет
\begin{figure}[H]
    \centering
    \includegraphics[width=1.02\columnwidth]{chrome_RcedAI0rkr}
    \caption{}
    \end{figure}
\end{multicols}
%\end{frenchspacing}
\end{document}
